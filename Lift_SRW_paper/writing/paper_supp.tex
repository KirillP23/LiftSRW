\section{Supplement to "Estimating Graphlets via Lifting"}

\subsection{Proof of Prop. \ref{prop:unbiased}.}
    \begin{proof}
    Let $\phi_i$ be as defined in \eqref{phi:ord}, \eqref{phi:shot}.
    For both estimators, because of the form of \eqref{eq:ord_est} and \eqref{eq:shot_est}, if a single term $\phi_i$ is unbiased then $\hat N_m$ is as well.
    Let us begin with $\hat N_{O,m}$, by considering a draw from the lifting process, $A = [v_1,\ldots,v_k]$ which induces the $k$-subgraph, $G|A$.
    By the definition of $\tilde \pi$,
	\begin{multline*}
	    \E \left( \phi_{O,1} \right) = \sum_{A \in V_G^k} \tilde \pi(A) \left( \frac{\ind(T(A) \sim H_m)}{\co(T(A)) \tilde \pi(A)} \right) \\
	    = \sum_{T \in \cV_k} \sum_{A \in V_G^k : T(A) = T} \frac{\ind(T \sim H_m)}{\co(T)} = \sum_{T \in \cV_k} \ind(T \sim H_m) = N_m.  
	\end{multline*}
	Hence, the $\hat N_{O,m}$ is unbiased.
	Consider the shotgun estimator, $\hat N_{S,m}$,
	\begin{multline*}
	    \E \left( \phi_{S,1} \right) = \sum_{B \in V_G^{k-1}} \tilde \pi(B) \sum_{u \in \cN_v(B)} \left( \frac{\ind(G|B \cup \{u\} \sim H_m)}{\co(H_m) \tilde \pi(B)} \right) \\
	    = \sum_{T \in \cV_k} \sum_{B \in V_G^{k-1}} \ind(G|B \cup \{u\} = T, u \in \cN_v(B)) \frac{\ind(T \sim H_m)}{\co(T)} \\
	    = \sum_{T \in \cV_k} \ind(T \sim H_m) = N_m.  
	\end{multline*}
    Hence, the shotgun estimator is unbiased as well.
	\end{proof}


\subsection{Proof of Theorem \ref{thm:var_bd}.}
We can bound the variance in \eqref{eq:variation.ind} by the second moment, which is bounded by,
\begin{equation*}
    \E \phi_1^2 \leq \E \phi_1\max{\phi_1} = N_m(G)\max{\phi_1}.
\end{equation*}
Seeking to control the the maximum of $\phi_1$, we see that,
\begin{equation*}
    \max_T \frac{1}{\pi_U(T)} \le \max_A \frac{1}{|\co(T)| \tilde \pi(A)} \le 
    \max\frac{\prod_{r=1}^{k-1} (d_1+\ldots+d_r)}{|\co(H_m)|\pi_1(d_1)},
\end{equation*}
\begin{equation*}
    \max_B \frac{|\cN_v(B)|}{|\co(H_m)|\tilde\pi(B)} \le 
    \max\frac{\prod_{r=1}^{k-1} (d_1+\ldots+d_r)}{|\co(H_m)|\pi_1(d_1)}.
\end{equation*}
Thus, we can construct a bound on $V_m^{\independent}(\phi_1)$.

\subsection{Proof of Theorem \ref{thm:cor_bd}}

Let $\phi_i$ be as defined in \eqref{phi:ord}, \eqref{phi:shot} or \eqref{phi:unord}.
Given two starting vertices $v_i$ and $v_j$ of the lifting process, notice that random variables $\phi_i|v_i$ and $\phi_j|v_j$ are independent.
% Given large $n$, the covariance term from \eqref{eq:variation} can be approximated as follows
% \begin{multline*}
%     \frac{2}{n^2} \sum_{i < j} \Cov_{\pi_k}\left(\phi(T_i), \phi(T_j)\right) \approx
%     \frac{2}{n} \big( \Cov_{\pi_k}\left(\phi(T_t), \phi(T_{t+1})\right) \\
%     + \Cov_{\pi_k}\left(\phi(T_t), \phi(T_{t+2})\right) + \ldots\big).
% \end{multline*}
Therefore
\begin{multline*}
    \E\left(\phi_i \phi_{i+1})\right)=
    \E_{\pi_1(v_i)\times \pi_1(v_{i+1})} \E\left(\phi_i \phi_{i+1}| v_i, v_{i+1}\right) =\\
     \E_{\pi_1(v_i)\times \pi_1(v_{i+1})} \left(\E\left(\phi_i|v_i\right) \E\left(\phi_{i+1}|v_{i+1}\right)\right).
\end{multline*}

Using the equation above, we can bound the covariance of $\phi_i$ and $\phi_{i+1}$ with basic inequalities:
\begin{align*}
    |\Cov\left(\phi_i, \phi_{i+1}\right)| \leq & \\
    \sum_{x_1,x_2\in V_G}  \E(\phi_i|v_i = & x_1)  \E\left(\phi_{i+1}|v_{i+1}=x_2\right)\\
    &\left\vert \P(v_i=x_1,v_{i+1}=x_2) - \pi_1(x_1)\pi_1(x_2)\right\vert\leq \\
    \max_{x_2\in V_G}  \E(\phi_{i+1}|v_{i+1} & =x_2) \sum_{x_1} \E\left(\phi_i|v_i = x_1\right)\\
    \max_{x_1} & \sum_{x_2}  \left\vert \P(v_i=x_1,v_{i+1}=x_2) - \pi(x_1)\pi(x_2)\right\vert = \\
      2\gamma_{G_V}(h) \max_{x_2} &\ \E\left(\phi_{i+1}|v_{i+1}=x_2\right) \sum_{x_1} \E\left(\phi_i|v_i = x_1\right),
\end{align*}

where $\gamma_{G_V}(h)$ is the mixing coefficient from \eqref{eq:gamma} for the random walk on vertices.
Next, estimate factors from the RHS as follows:
\begin{multline}
    \sum_{x}\E\left(\phi|v = x\right) \leq
    \max_{x}\frac{1}{\pi(x)} \sum_{x}\E\left(\phi|v = x\right) \pi(x)\leq \\
    2|E_G| N_m(G).
\end{multline}

For $\max_x \E\left(\phi|v=x\right)$, consider the expressions for $\phi$ from \eqref{phi:ord}, \eqref{phi:shot} or \eqref{phi:unord}.

Using notation $D = \prod_{r=2}^{k-1} (\Delta_1 +\ldots + \Delta_r)$, for the Ordered Lift estimator,
\begin{multline*}
    \max_x \E\left(\phi_O|v=x\right) \le \max_x \sum_{A} \frac{\P(A|v=x)}{\tilde\pi(A)} \le \\
    \max_x \frac{|\{A\mid A[1]=x\}|}{\pi(x)} \le 2|E_G|D.
\end{multline*}
For the Shotgun Lift estimator,
\begin{multline*}
    \max_x \E\left(\phi_S|v=x\right) \le
    \max_x \sum_B |\cN_v(B)|\frac{\P(B|v=x)}{\tilde\pi(B)} \leq \\
    \max_x \frac{|\cN_v(B)| |\{B\mid B[1]=x\}|}{\pi(x)} \leq
    2|E_G| D,
\end{multline*}
For the Unordered Lift estimator,
\begin{multline*}
    \max_x \E\left(\phi_U|v=x\right) \le
    \max_x \sum_T \frac{\P(T|v=x)}{\pi_U(T)} \leq \\
    \max_x \frac{|\{T\mid x\in V_T\}|}{\pi(x)} \leq
    2|E_G| D.
\end{multline*}

Combining the results, we get the desired bound.
